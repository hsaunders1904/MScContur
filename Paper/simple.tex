\documentclass[floatfix]{article}  
\pagestyle{plain}

\usepackage{amsmath,amssymb}
\usepackage{dsfont}
\usepackage{graphicx} %loads the graphicx.sty package 
\usepackage{epstopdf} %loads the epstopdf.sty package 
\usepackage{slashed}
\usepackage{color}

\usepackage{graphicx,longtable,tocloft,color}
\usepackage{sidecap}
\usepackage{subfig}
\usepackage{xspace}
\usepackage{mydefs}
\usepackage{cite}
\usepackage{lineno}
\usepackage{multirow}
\usepackage{hyperref}
%\usepackage[left=2cm,right=2cm,top=2cm,bottom=2cm]{geometry}
%\tolerance = 1000
%\parindent 0cm
%\parskip 2mm
%
%
%for draft only:
%\linenumbers
\usepackage{array}

\title{Constraining models of new physics with collider measurements of Standard Model signatures}

\author{Jonathan M. Butterworth$^1$, David Grellscheid$^2$,\\ Michael Kr\"amer$^3$, David Yallup$^1$\\
\it $^1$Department of Physics and Astronomy, University College London,\\ \it Gower St., London, WC1E 6BT, UK\\
\it $^2$IPPP, Department of Physics,\\\it Durham University, DH1 3LE, United Kingdom\\ \it $^3$Institute for Theoretical Particle Physics and Cosmology, \\ \it RWTH Aachen University, Aachen, Germany}


\begin{document}

\maketitle 

\begin{abstract}
The Large Hadron Collider is probing physics in a new kinematic region, at energies around and above the 
electroweak symmetry-breaing scale. Many searches for signatures of physics beyond the Standard Model have 
been performed, and extensive measurements of various final states have also been 
carried out. At time of writing, the results are generally in accord with Standard Model expectations, 
notwithstanding a few currently marginal anomalies. Whether physics beyond the Standard Model is discovered 
or not, there is a need to extract the clearest and most generic information about physics in this new energy regime,
an imperative which will grow with integrated luminosity. The framework of so-called 'simplified models', coupled
with the fact that model-depedence in many LHC measurements is minimal, allows strong constraints on new physics 
models to be placed, using existing and future measurements. We demontrate the power of this approach using a competitive 
simplified Dark Matter model. The method is highly scaleable to other models of further measurements, as they appear.
\end{abstract}


\section{Introduction}
\label{sec:intro}

With the discovery of the Higgs boson~\cite{Aad:2012tfa,Chatrchyan:2012ufa}, 
the first data-taking period of the experiments at the 
Large Hadron Collider demonstrated that the understanding of electroweak symmetry-breaking within
the Standard Model (SM) is broadly correct, and thus that the theory is potentially valid well above the
TeV scale. Many precision measurements of jets, charged leptons, and other of final states, 
have been published, reaching into this new kinematic domain. The predictions of the SM are 
generally in agreement with the data, while the many dedicated searches for physics beyond the SM
have excluded a wide range of possible scenarios. Nevertheless, there are many reasons to be confident that
physics beyond the Standard Model (BSM) exists; examples include the graiational evidence for Dark Matter, the large 
preponderance of matter over antimatter in the universe, and the existence of gravity itself. None of 
these can be easily accommodated within known Standard Model phenomenology. 

This motivates a continued campaign to make precise measurements and calculations at higher energies and 
luminosities, and to exploit these measurements to narrow down the class of viable models of new physics, 
hopefully shedding light on the correct new theory, or at least on the energy scale at which
new physics might be observed at future experiments.

In this paper we exploit three important developments to survey existing measurements and set 
limits on new physics. 

\begin{enumerate}
\item
SM predictions for differential and exclusive, or semi-exclusive, final states are made using sophisticated 
calculational software, often embedded in Monte Carlo generators capable of simulating full, realistic final 
states~\cite{Buckley:2011ms}. These generators now incorporate higher-order processes
matched to logarithmic parton showers and successful models of soft physics such as hadronisation and
underlying event. They are also capable of importing new physics models into this framework, thus allowing
rapid predictions of their impact on a wide variety of final states simultaneously. 
In this paper we make extensive use of these capabilities within Herwig~\cite{Bahr:2008pv}.
\item
As many favoured BSM scenarios have been excluded, there has been a move toward 
``simplified models'' of new physics~\cite{Alves:2011wf,Abercrombie:2015wmb}, which aim to be as generic as possible, 
while being internally consistent. The philosophy is similar to an ``effective lagrangian'' approach in which effective
couplings are introduced to describe new physics, but is more powerful as such simplified models also
include new particles, and thus can remain useful up to and beyond the scale of new physics - a region 
potentially probed by LHC measurements.
\item
The precision measurements from the LHC have mostly been made in a manner which minimises their model-dependence. 
That is, they are defined in terms of final-state signatures in fiducial regions well matched to the
acceptance of the detector. The majority of such measurements are readily available for analysis and
comparison in the Rivet library~\cite{Buckley:2010ar}. 
\end{enumerate}

These three developments together make it possible to efficiently 
bring the power of a very wide range of data to bear on the search for new physics. While such a generic approach is
unlikely to compete in terms of speed and sensitivity with a search optimisied for a specific theory, the breadth
of potential signatures and models which can be covered makes it a powerful complementary approach. Any theory seeking
to explain a new signature or anomaly in the data may have consequences for other final states which should be checked 
against data this way. If no BSM physics emerges, a model-dependent and systematic approach becomes mandatory.

The outline of the paper is as follows. In the next section, we motivate and describe the 
simplified model we consider as a first demonstration, and its implementation via Herwig. 
In Section~\ref{sec:measurements} we introduce the measurements that we will use, and their implementation in Rivet. 
Section~\ref{sec:kinematics} discusses the differential cross sections in which the impact of these models would be 
most apparent. In Section~\ref{sec:limits} this impact is translated into limits on the model parameters.

\section{Simplified Model}\label{sec:models}

... for Dark Matter

Describe the model and its implementation via feynrules and herwig.

\cite{Kahlhoefer:2015bea}

\section{Measurements}\label{sec:measurements}

To be useful in our approach, measurements must be made in as model-independent fashion as possible. 
Cross sections should be measured in a kinematic region closely matching the detector acceptance - commonly called 
'fiducial cross sections' - to avoid extrapolation into unmeasured regions, since such extrapolations must make 
theoretical assumptions; usual that the SM is valid. The measurements should generally be made in terms of observable final
state particles (e.g. leptons, photons) or objects constructed from them (e.g. hadronic jets, missing energy) 
rather than assumed intermediate states ($W, Z, H$, top). Finally, differential measurements are most useful, as features
in the shapes of distributions are a more sensitive test than simple event rates - especially when there are
highly-correlated systematic experimental uncertainties, such as those on the integrated luminosity, or the jet energy scale.

One feature noted in several cases is that missing transverse energy (\MET) is 
explicity assumed to be the same as neutrino transverse energy. In fact, in a BSM study, missing energy can also 
arise from other sources (for example, Dark Matter production!) and so it is important that the result is treated in such a 
way that this sensitivity is correctly estimated. The measurements are typically corrected back to total \MET, 
or to the assumed neutrino $p_T$, in the experimental analysis, using a simulated SM event sample which has been shown to describe
the data well. This involves an extrapolation into the forward 
region where transverse energy is unmeasured; however, unless a BSM particle enters this region, the error made is 
negligible. This means that as long as fiducial acceptance cut is made on BSM particles counting toward \MET (to ensure that large contributions to \MET 
from invisible particles outside the detector acceptance are excluded) such analyses can be 
used\footnote{Of greater consquence, but easier to fix, is the fact that several rivet methods explicitly calculated \MET
from neutrinos found in the true event record, rather than as the negative of the visible particles in the event. These 
routines were modified as a part of this work, and are all fixed in release 2.5.X.}

The measurements we consider fall into five loose and independent classes.
\begin{enumerate}
\item
Jets: event topologies with any number of jets but no missing energy, leptons, or photons. In this category there
are important measurements from both ATLAS and CMS, many of which have existing Rivet analyses. We make use of 
the highest integrated-luminosity inclusive~\cite{Aad:2014vwa,Chatrchyan:2014gia} dijet~\cite{Aad:2013tea,Aad:2014pua} 
and three-jet~\cite{Aad:2014rma} %{\bf add cms 3jet here? Needs a rivet routine.} 
measurements made in 7~TeV collisions, as well as the jet mass 
measurement from CMS~\cite{Chatrchyan:2013vbb}.
Unfortunately results from 8~TeV collisions are rarer, and the only one we can use currently is the four-jet 
measurement from ATLAS~\cite{Aad:2015nda}.
\item
Electroweak: events with leptons, with or without missing energy or photons. The high-statistics $W+$jet and $Z+$jet measurements from ATLAS~\cite{Aad:2014qxa,Aad:2013ysa} 
and CMS~\cite{Khachatryan:2014uva}, are used.
We also use the ATLAS $ZZ$ and $W/Z+\gamma$ analyses~\cite{Aad:2012awa,Aad:2013izg}.
\item
Missing energy, possibly with jets but no leptons or photons. There are no fully-correct particle-level distributions available. 
However the ATLAS search for supersymmetry, with jets and missing energy~\cite{Aad:2012fqa} does have a rivet routine and can be used approximately (I hope)
\item
Isolated photons, with or without missing energy, but no leptons.
\item
Signatures specifically based on top quark or Higgs candidates. These will in principle overlap with
the previous categories depending on decay mode, but we discuss them seperately.
\end{enumerate}

The choice of which measurements to include is driven mainly by the availability of particle-level differential fiducial cross sections 
implemented in rivet routines.

\section{Statistical Method}

Because of correlations between uncertainties, both statistical and systematic, ...

Explain the assumptions and statistical approach.


\section{Comparison to Data}\label{sec:kinematics}


\section{Limits}\label{sec:limits}

\section{Conclusions}\label{sec:conclusions}

\section*{Acknowledgments}



\bibliographystyle{h-physrev4}
\bibliography{simple}

\end{document}


